% $Id: introduction.tex 1784 2012-04-27 23:29:31Z nicolas.cardozo $
% !TEX root = main.tex

\chapter{Introduction}
\label{cha:introduction}
The typewriter interface, through the keyboard, has been the reigning interface to materialize programs into code since the beginning of programming.

Interface designs lack what is needed to support every user. Naturally, users are not created equally.
Prior work in the user experience and human-interface integration show reports in which technology and its designs can discriminate against users due to several factors,
such as different physical or cognitive capacities, race and gender \cite{ko23}.
Moreover, programming has a steep learning curve and a high entry level \cite{bosse17}.
Different approaches to computing are needed to lower the entry threshold and make computing more approachable to everyone.

Current programming languages are based on the same principles that we have used since the 50s.
The same principles that were designed to be used with a typewriter and a flat surface in which to write.
To lower the entry threshold of programming and computing, more approachable interfaces and methods are required.
New fields in computing are emerging with different intents but still the same interfaces \cite{hongji16}.
In order to explore new methods of computing, we must look further ahead than the tools we, into how we approach code in general.

To do so, we must shift attention to other mediums that can be used for computation, yet serve a complete different purpose.
Such path could lie into the realm of art. Using coding tools to create art is not a new concept.
Many artists today embrace code in their practice, creating tools and environments to create interesting pieces of work. 
For example, Sonic Pi is an entire coding suite built over SuperCollider sound synthesis server that allows users to create music using code with a superset of Ruby 
tailored for live coding performances and initially conceived as a programming teaching tool for children in England \cite{aaron16}.

\sculpt is an artistic sandbox with rules that ensure correct computation, shifting the focus away from computation, yet relying heavily on it to create physical sculptures that, by design, compile to valid programs.

\endinput