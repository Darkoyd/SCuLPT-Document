\section{Methodology}
\label{sec:methodology}

Artlang as a whole is in fact two interconnected languages, one being the visual language and specification that defines the components from a strictly physical standpoint and the other being the programming language that is used to define the behavior of each component.
For the development of Artlang, we have followed a user-centered design approach, creating a visual language that is both usable and accessible to a wide range of users.
Artlang's focus on usability and accessibility is displayed by using simple but distinctive shapes and forms for each component on the language.
This first design choice allows for users to easily identify and differentiate between different components of the language, making it trivial to convey meaning behind each shape while allowing people with any background to recognize each component foregoing knowledge, language or capacity barriers that arise from conventional programming languages.

Regarding the programming language, Artlang is designed to be a Turing complete language. 
By building structures and following the intuitive rules of the language, users can create any program they desire. 
The language is designed to be simple and easy to understand, with a focus on the visual representation of the program rather than the code itself.
Code block are physically assembled together to create a program, with each block representing a different function or operation. Block are designed such that their connections are intuitive and easy to understand, with each block having a set of features that must be connected to other blocks in order to function correctly.
This allows users to create programs without the need to understand complex syntax or semantics, making it accessible to users with little to no programming experience.


Regarding the language's aesthetics, Artlang is designed to be an artistic sandbox therefore the rules and specification for how shapes should look and feel are very loose. This allows users to tinker with the components, suiting them for their needs and desires.
The only rules that must be followed only consider are component connections and a set of features each block requires. Final shape, size, material, color or texture are completely open for anyone to modify without any impact on the program.
This provides a sense of freedom and creativity that is not present in other programming languages, allowing users to express themselves through their code and create unique and personal pieces of art.
This also implies that the language can be fitted to almost anyone's needs, allowing accessibility and customization for users with different physical or cognitive capacities.

Artlang current implementation features small 3D printed components with varying colors.
The printed pieces follow the standard Artlang implementation with no modifications to the shapes.
The components are connected using magnets and screw elbow joints, allowing for easy assembly and disassembly of the components, making it easy to create and modify programs on the fly.
\endinput