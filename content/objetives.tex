\section{Objectives}
\label{sec:objectives}
In the creation of \sculpt we posit two objectives.
First, we require to build a fully capable physical programming language addressing the concerns described previously about the state of programming, computation and programming languages.
\sculpt is designed to be a visual and physical oriented programming language, with a heavy focus on the ease of use, computational correctness and aesthetics.
\sculpt is designed to be a Turing complete language with a very small instruction set, allowing users to create any program they desire.

Second, we aim to evaluate the language in two main aspects.
First, we intend to evaluate the ease of the language to foster computational thinking without the intrinsic difficulties of current programming languages imposed by the current interfaces.
This is done by evaluating the performance of users with different backgrounds and levels of experience in programming, from novices to expert programmers, in solving small computational problems using \sculpt
and their overall experience with the language.
Second, we evaluate the properties of \sculpt as a paradigm that breaks existing preconceptions about programming languages and moves beyond 
the typewriter interface by leveraging artistic creativity as an alternative.
This is done by letting users create pieces freely using \sculpt, allowing them to explore the language and its components without any pressure to complete any tasks.
This allows us to evaluate the overall experience of using \sculpt as an artistic tool, as well as any suggestions for improvement from an aesthetic lens.
\endinput