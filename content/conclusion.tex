
\section{Conclusion and Future Work}
\label{sec:conclusion}
\sculpt is a visual and physical oriented programming language that aims to lower the entry threshold to programming by leveraging artistic creativity as an alternative to the traditional typewriter interface.
We have presented the design and implementation of \sculpt, a language that allows users to create physical sculptures that compile to valid programs as well as a functional runtime to run and test programs before assembling the final sculpture.
The design choices in \sculpt make it possible for anyone to create programs without the need for prior programming knowledge, making it accessible and modifiable to a wide range of users.

While the validations done so far are only preliminary, they show promising results in terms of usability and accessibility, while resulting in interesting pieces of art.

\subsection{Future Work}
\sculpt is still in its early stages and there are several areas for future work.
First, we plan to conduct more extensive user studies to gather more data on the usability and accessibility of \sculpt, as well as the overall experience of using the language.
Second, we plan to explore the use of \sculpt in different contexts and environments, such as in schools or workshops, to evaluate its effectiveness as a potential teaching tool.
Third, we intend to enhance \sculpter to support more complex features, such as better error handling, a language server, syntax highlighting, and code completion.
Also enhance \sculpter with interesting features aimed at artists, such as a visualizer to see the program as a sculpture before assembling it, exporting the sculpture's blocks as 3D models for 3D printing and assembling and sculpture scanner to parse a sculpture and generate the corresponding program using images.
Finally, we plan to explore the use of \sculpt in different artistic contexts, such as in interactive installations or performances, to evaluate its potential as an artistic tool.

\endinput

